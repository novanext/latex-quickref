\documentclass[fleqn, 12pt]{iopart-mod}

\RequirePackage[load-configurations=abbreviations, mode=text]{siunitx}
\DeclareSIUnit{\micm}{\micro\metre}
\usepackage{xcolor}
\definecolor{darkblue}{rgb}{0,0,0.5} 
\definecolor{red}{rgb}{1,0,0} 
\usepackage{graphicx}
\usepackage[intlimits]{amsmath}
\usepackage{url}
\usepackage{doi}
\usepackage{pslatex}
\usepackage{booktabs}
\usepackage{ragged2e}

\begin{document}

\title{Proper use of \LaTeX{}}

\author{Johan B.~C. Engelen}
\ead{j.b.c.engelen@alumnus.utwente.nl}

\address{University of Twente, Enschede, the Netherlands} 
        
\begin{abstract}
How to properly  write scientific text using \LaTeX{}. Many people make many typographical mistakes, even journals make them. And because I am quite pedantic about this, here is a list of how it should be done. Unfortunately for you, it is not up for debate, it is just the way it is written below. 
\emph{Authors of scientific literature are likely to have no clue about all this!}
\emph{Editors of scientific literature are (arguably a little less) likely to have no clue either!} 
Typesetters of scientific literature know something, but they too make mistakes; see \emph{e.g.} J. Micromech. Microeng., with an italic `mu' in micrometer. 

Following the rules here, I think you'll find it much easier to read math. Mathematical typography requires attention to detail. 
You should understand the difference in meaning between $\mu_p = \left(\frac{T}{100}\right)^p$ and $\mu_\text{p} = \left(\frac{T}{100}\right)^p$.
Perhaps then you'll appreciate the beauty of written mathematical language.

Have fun!

\vspace{5em}
For a less agitated (and better, but longer) text on scientific typesetting, see \cite{Beccari1997}.
\end{abstract}

\maketitle

%#############################################################
\RaggedRight

\section{Packages}
Have a look at the preamble of this document. Useful packages are: \verb|siunitx| for writing units, \verb|booktabs| for making nice tables, \verb|amsmath| mostly for the \verb|\text| command.

For IOP journals, use the modified LaTeX package used by this package, in order to use \verb|siunitx| and \verb|amsmath|.

\section{Units}

\textbf{Units} should be \textbf{upright}, not italic. Why? Because in most contexts, $10 cm^2$ means ten times the speed of light times variable $m$ squared. Ten centimeter squared should be typeset as \SI{10}{cm^2}, written in \LaTeX{} as \verb|10~cm$^2$|, or (better) using the \verb|siunitx| package: \verb|\SI{10}{cm^2}|. Note the thin space between number and unit.
The package documentation of \verb|siunitx| is well worth reading.

Micrometer is abbreviated by \si{\micm}, note that the `mu' is upright! $\mu$m is wrong and means something like permeability times meter? Compare with an acceleration of $10g$~\si{N.m^{-2}}, where $g$ is the standard gravity; and $10$~\si{g.N.m^{-2}}, meaning 10 grams Newton per meter squared. An interesting unit is the `kilo Watt hour' unit \si{\kWh}: \SI{11}{\kWh} and \SI{11}{\kWh\per\metre} (note the behaviour of \verb|siunitx|: (absence of) thin space between `W' and `h').

In tables, the units should \textbf{not be between square brackets}. Correct usage of the brackets: $\left[ F \right] = \si{\newton}$, so you hardly ever want to use that. If you want, you can put the units between normal parens ().

\section{Subscripts}
\textbf{Subscripts} in math should in most of \emph{our} cases be \textbf{upright}. Why? Because a subscript in italic has a mathematical meaning; an upright subscript is just simple text, meaning a word or abbreviation of something. To write an upright subscript, use \verb|\text|, e.g. \verb|$k_\text{eff}$|, $k_\text{eff}$.

\begin{tabular}{@{}ll@{}}
\toprule
Symbols   &  Meaning \\
\cmidrule(r){1-1}\cmidrule(l){2-2}
$E_x$            & electric field in $x$ direction \\
$E_\text{plate}$ & electric field due to some charged plate \\
$k_{eff}$        & $k$ with indices $e$, $f$, and $f$. So e.g. $k_{122}$ \\
$k_\text{eff}$   & effective $k$ (e.g. an effective spring stiffness) \\
$n_i$            & $n$ with index $i$, e.g. $\displaystyle\sum_{i=1}^{10} {n_i}$ \\
$n_\text{i}$     & $n$ with subscript abbreviation `i', perhaps intrinsic carrier concentration \\
$m_\text{e}$     & electron mass \\
$k_B$            & some $k$ having to do with a magnetic field $B$? \\
$k_\text{B}$     & = \SI{1.3806504(24)e-23}{\joule\per\kelvin} \\
$E_\text{xmax}$  & yuk! \\
$E_{x,\text{max}}$  & neat! \\
$\displaystyle\int_{all space}$ & lots of indices! \\
$\displaystyle\int_\text{all space}$ & superb \\
\bottomrule
\end{tabular}

\section{Misc math stuff}

\begin{tabular}{@{}ll@{}}
\toprule
Symbols   &  Meaning \\
\cmidrule(r){1-1}\cmidrule(l){2-2}
$cos(2\pi)$      & $c$ times $o$ times $s(2\pi)$ (probably $s$ is a function) \\ 
$\cos(2\pi)$     & = 1 \\
$\cos^{-1}(2\pi)$     & = 1 \\
$\arccos(2\pi)$     & argument $2\pi$ is outside the domain of the inverse cosine function \\
$exp(\ldots)$    & $e$ times $x$ times \ldots \\
$\exp(\ldots)$    & $e^{\ldots}$, this is probably what you meant \\
\bottomrule
\end{tabular}

\section{Approximately, proportional to, plus-minus}
If something (for example, a measurement error or actuation range) ranges from \SI{-50}{\micm} to \SI{+50}{\micm}, you write that as \verb|\SI{+-50}{\micm}|, \SI{+-50}{\micm}. If something is approximately \SI{50}{\micm}, you write \verb|$\sim$\SI{50}{\micm}|, $\sim$\SI{50}{\micm}, or \verb|$E \sim 5$|, $E \sim 5$, or \verb|$E \approx 5$|, $E \approx 5$. If $E$ is proportional to $T$, you write \verb|$E \propto T$|, $E \propto T$.

\section{Quotes}
Use \textbf{double quotes} for \textbf{real quotations}, i.e. text that has actually been said or written. ``For example,'' Johan said. 
Use \textbf{single quotes} for \textbf{`strange'} words. 
Note that the start and end quotes are different! Use \verb|`| at the start, and \verb|'| at the end.

%#####################################
\RaggedRight
\bibliographystyle{bst/johan-thesis2}
\bibliography{paperbase}

\end{document}

